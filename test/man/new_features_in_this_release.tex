% Generated by html2tex: Version 2.7 of February 6, 2012.
% Written by F.J. Faase.  http://www.iwriteiam.nl/
% Adapted by J.K. Crook.  http://audacityteam.org/

% html: Beginning of file: `new_features_in_this_release.html'
% DOCTYPE html
% [if IE 6]><link rel="stylesheet" href="../m/skins/monobook/ie60fixes.css/303.css" media="screen"/><![endif]
% [if IE 7]><link rel="stylesheet" href="../m/skins/monobook/ie70fixes.css/303.css" media="screen"/><![endif]
																					
\chapter{New features in this release}

\label{f0}																																	%  start content 
					

% <br/ lang="en">
\includegraphics[max width=\linewidth]{../m/images/8/88/audacity_logo_signika_512_transparent.png}\footnote{See URL http://audacityteam.org/}\newline\textbf{This page is an overview of the key new functionality that has been introduced in Audacity 2.2.0}
\begin{itemize}
\item  Details of all the major changes since 2.1.3 can be found in Release Notes 2.2.0.
\end{itemize}


% <br/ rel="nofollow">


% <br/ rel="nofollow">

\section{New Logo}


The logo has been given a refresh, and now uses a sans-serif font and a flatter style.

\includegraphics[max width=\linewidth]{../m/images/8/8c/audacity_logo_whitebg.png}\footnote{See URL http://audacityteam.org/}

% <br/ /="">

\section{Themes}


Audacity now comes supplied with four pre-configured, user-selectable, themes.  This enables you to choose the look and feel you prefer for Audacity's interface. see the Themes page for details.
\begin{itemize}
\item \textbf{Light} theme: this is a light theme loosely based on the look and feel of earlier Audacity versions, but given a contemporary twist with more modern-looking buttons and icons. 
\item \textbf{Dark} theme: created by the Dark Audacity project.\footnote{See URL http://www.darkaudacity.com/} This is similar to the Light theme, with the same buttons and icons, but given a dark twist.
\item \textbf{Classic} theme: The one you know and loved. This theme is a re-creation of the look and feel of earlier Audacity versions. 
\item \textbf{High Contrast} theme: some users with poor eyesight benefit from a high contrast that is 'eye-popping' for most people.
\end{itemize}


% <br/ rel="nofollow">

% <tbody rel="nofollow">
\* \* \* \* \includegraphics[max width=\linewidth]{../m/images/5/51/theme_light.png}
\* \* \* \* \includegraphics[max width=\linewidth]{../m/images/e/e3/theme_dark.png}
\* \* \* \* 
\textbf{Light} theme
\* \* \* \* 
\textbf{Dark} theme
% </tbody valign="top">

% <tbody valign="top">

\* \* \* \* \includegraphics[max width=\linewidth]{../m/images/b/bc/theme_classic.png}
\* \* \* \* \includegraphics[max width=\linewidth]{../m/images/c/c3/theme_hi_contrast.png}
\* \* \* \* 
\textbf{Classic} theme
\* \* \* \* 
\textbf{High Contrast} theme
% </tbody valign="top">



% <br/ valign="top">
The theme to use can be chosen at Edit \mbox{$>$} Preferences \mbox{$>$} Interface.  
\begin{itemize}
\item  In addition to the four pre-configured themes there is also a Custom theme.  
\end{itemize}
 By default the custom theme looks the same as Classic theme - but, if you have the right programming skills and tools, you can use this template to create your own theme.  Instructions for how to do this may be found in the Audacity Wiki.


% <br/ rel="nofollow">

\section{MIDI (and Allegro) Playback}


Playback of MIDI (and Allegro) files imported into Note Tracks is now available.  Please see the Note Tracks page for more details.

This should just work on Windows but for playback on Mac and Linux additional software may be required, see this section on the Playing and Recording page.

But note that there will no use of the Playback meter while Note tracks are played. 


% <br/ title="Meter Toolbar">

\section{Stem Plots}


There is a new entry in the Tracks Preferences for \textbf{Display samples}.  This setting changes how Waveform and Waveform dB views are displayed.  It only affects the appearance of the waveform when you are so far zoomed in that you can see the individual sample dots.  At lower zoom levels it makes no difference.  
\begin{itemize}
\item \textbf{Stem plot:} This is the default setting which draws a vertical line from the track center line to the sample dot, giving a clearer impression of the relative amplitude of the samples. As seen in the images below, when zoomed out close to the minimum for a stem plot, the horizontal distance between sample dots may be more uneven than seen with the connect dots default.   
\item \textbf{Connect dots:} This is alternative setting yields a waveform where each sample dot is connected to the next sample by a line drawn between them.  
\end{itemize}
\includegraphics[max width=\linewidth]{../m/images/e/ef/connect_dots_stem_plot_examples.png}??Uneven spacing is due to \texttt{{}"{}}aliasing\texttt{{}"{}}, but zooming in further will equalize the spacing whether choosing Connect dots or Stem plot.


% <br/ class="note">

\section{Menu reorganization}


We have made the Menus shorter and clearer than in previous Audacity versions. The menus have been simplified without losing functionality. The most commonly used functions are found in the top levels of the menus. The functions moved down into lower submenus are better organized.

This is not just a rearrangement.  We also added new menu items to make the layout more logical.  There are new menu items for exporting as MP3 or WAV. Previously you had to export audio, and then choose the format. You still can do that, but these new items are there for convenience.
One of the long standing bug-bears with Audacity is the distinction between \&lsquo;Save\&rsquo; and \&lsquo;Export\&rsquo;. People expect to be able to open a WAV audio file, edit it in Audacity and then click save. That is not how Audacity works. Audacity needs audio in its own unique format to work on it. So Audacity converts when you 'Open' a WAV file and converts back when you 'Save'. In an attempt to make this clearer Audacity uses the word 'Import' for opening a file like WAV and 'Export' for converting and saving in a format like WAV. Open and Save are reserved for Audacity's own project format.

Now the \&lsquo;Export\&rsquo; options are under the \&lsquo;Save Other\&rsquo; menu item, where people trying to save audio as an MP3 or WAV file are more likely to find them. 


% <br/ class="note">

\section{The Extended Menu bar}


There are two new additional menus that are hidden by default.  They can be turned on at View \mbox{$>$} Extra Menus (on/off) or the Interface pane of Edit \mbox{$>$} Preferences.

These extra menus have many extra less frequently used commands.  They are particularly useful to VI users, but normally-sighted users may find them useful too.
\textit{\textbf{Image of the Extended Menu bar as it appears on Windows}}\includegraphics[max width=\linewidth]{../m/images/0/0d/menu_toolbar_extra.png}??
\textit{Click, or hover, on either of the \textbf{Ext-} items at the end of the image to read about those menu entries.}
% <tbody class="prettytablerows">

Menu
What you'll find there
Ext-Bar
The Ext-Bar menu provides access to Toolbar operations that are not available in the default Audacity menus. These will be of most interest to visually impaired users or those who have difficulty using the mouse.

Shortcuts can be assigned to these commands if required.
Ext-Command
The Ext-Command menu provides access to extra commands for track focus and movement of the editing or playback cursor that are not available in the default Audacity menus. These will be of most interest to visually impaired users or those who have difficulty using the mouse.

Shortcuts can be assigned to these commands if required.
% </tbody title="Keyboard Preferences">

If you only require regular access to a small set of these commands you can set shortcuts for them and leave the extra menus hidden.


% <br/ class="note">

\section{Appended recording on the same track in now the default}


From Audacity 2.2.0 onward the default recording mode has changed so that when you click the Record button \includegraphics[max width=\linewidth]{../m/images/e/e8/record.png} on Transport Toolbar, or use the R, Audacity will record at the end of the currently selected (or only) track.

\subsection{To record on a new track}


If you hold the Shift button down the Record button in Transport Toolbar will temporarily change to \includegraphics[max width=\linewidth]{../m/images/7/72/record_new_track.png}. Then clicking on this modified Record button, or using the shortcut Shift + R will cause Audacity to create a new track and begin recording on that track from the current cursor position (or from the left edge of a region on the Timeline). 


% <br/ title="Timeline">

\section{Help buttons}


Many places in the user interface have had a help button \includegraphics[max width=\linewidth]{../m/images/d/d6/help_button.png} added. Examples are all the Preferences dialog panes, all the Effects, Generators and Analyzers and some error messages.

Clicking on that button in the dialog will link you to the appropriate page in the Manual.

Example: the Amplify effect.  Try clicking on the \texttt{{}"{}}\textbf{?}\texttt{{}"{}} at the bottom right of this image.\includegraphics[max width=\linewidth]{../m/images/6/6d/amplify.png}??


% <br/ /="">

\section{Standard and Full shortcut sets}


For Audacity 2.2.0 we have reduced the number of preset shortcuts in the application to a \texttt{{}"{}}\textbf{Standard}\texttt{{}"{}} set.  We did this to simplify the set of shortcuts somewhat and to provide greater flexibility for users who want to set their own custom shortcuts.

You can choose to revert to the full set of shortcuts that were in 2.1.3 and earlier by selecting \texttt{{}"{}}\textbf{Full}\texttt{{}"{}} from the dropdown menu accessed from the Defaults button in the Keyboard Preferences dialog.

You can use the Defaults button to switch between the two provided default sets of shortcuts at any time.\includegraphics[max width=\linewidth]{../m/images/f/f8/preferences_keyboard_with_default_dropdown_menu_trimmed.png}??

See Commands and Keyboard Shortcut Reference for more details.


% <br/ title="Keyboard Shortcut Reference">

\section{Selection Toolbar improvements}


There are now four available settings in the Selection and Audio Position Boxes in Selection Toolbar for the  manner in which the details of your selection are displayed:
\begin{itemize}
\item \textbf{Start and End of selection}: the start time and the end time of your selection \textit{(default setting)}
\item \textbf{Start and Length of selection}: the start time and the length of your selection
\item \textbf{Length and End of selection}: the length and the end time of your selection
\item \textbf{Length and Center of selection}: the length and the time at the center of your selection
\end{itemize}
\includegraphics[max width=\linewidth]{../m/images/c/c6/selection_toolbar_display_modes.png}??

% <br/ /="">

\section{New commands for using clips via the keyboard}


New commands, all of which interact with the clips on the focused track. As yet, there are no default shortcuts:
\begin{enumerate}
\item Transport \mbox{$>$} Cursor to \mbox{$>$} Previous clip boundary
\item Transport Cursor to \mbox{$>$} Next clip boundary
\item Select \mbox{$>$} Clip Boundaries \mbox{$>$} Previous clip boundary to cursor 
\item Select \mbox{$>$} Clip Boundaries \mbox{$>$} Cursor to next clip boundary 
\item Select \mbox{$>$} Clip Boundaries \mbox{$>$} Previous clip 
\item Select \mbox{$>$} Clip Boundaries \mbox{$>$} Next clip 
\end{enumerate}
There are also two new commands (Clip Left and Clip Right) for moving clips, which are available in the extended menus added in this release.

Note that these extended menus are \textit{\textbf{not}} turned on by default, to turn them on please see Menu Reference for how to reveal the extended menus.


% <br/ title="Menu Reference">

\section{Running out of disk space}


We have now provided an error trap for situations where you are running out of available disk space.

You will now see the error message:
\texttt{{}"{}}\textbf{Audacity failed to write to a file in \mbox{$<$}device\mbox{$>$}}\texttt{{}"{}}

This is particularly useful when recording as Audacity will stop recording when the error is trapped, preserving your recording up to that point.


% <br/ id="Running_out_of_disk_space">

\section{Additional new features}


This page \textbf{New features in this release - appendix} gives an overview of further new functionality that has been introduced in this release of Audacity.


% <br/ title="New features in this release - appendix">

\section{Links}


\textbf{\mbox{$>$}} Audacity Release Notes 2.2.0 \textit{- detailed release notes for this release of Audacity}%  
% NewPP limit report
% Cached time: 20171019181658
% Cache expiry: 86400
% Dynamic content: false
% CPU time usage: 0.129 seconds
% Real time usage: 0.139 seconds
% Preprocessor visited node count: 309/1000000
% Preprocessor generated node count: 824/1000000
% Post‐expand include size: 4813/2097152 bytes
% Template argument size: 3760/2097152 bytes
% Highest expansion depth: 4/40
% Expensive parser function count: 0/100
% 
%  
% Transclusion expansion time report (%,ms,calls,template)
% 100.00%   58.657      1 - -total
%   5.68%    3.330      2 - Template:Shortcut
%   5.31%    3.115      5 - Template:Note
%   4.53%    2.660      4 - Template:Button
%   4.10%    2.408      9 - Template:Menu
%   3.17%    1.862      1 - Template:Bh
%   2.63%    1.545      1 - Template:Hint
%   2.32%    1.358      1 - Template:Intro
%   2.16%    1.266      2 - Template:Ednote
% 
%  Saved in parser cache with key helpmediawiki:pcache:idhash:5555-0!*!0!!*!5!* and timestamp 20171019181658 and revision id 55934
%  

										%  end content 
																					%  end of the left (by default at least) column 
											
% html: End of file: `new_features_in_this_release.html'
